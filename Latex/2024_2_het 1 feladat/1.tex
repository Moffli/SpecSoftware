\documentclass[12pt, twoside]{article}
\usepackage{fancyhdr}
\usepackage{enumitem}
\usepackage{lipsum}

\setlength{\headheight}{15pt} % A "headheight too small" figyelmeztetés elkerülése

% Oldalstílus beállítása
\pagestyle{fancy}
\fancyhf{}
\fancyhead[LE]{\leftmark} % Section
\fancyhead[RO]{\rightmark} % Subsection
\fancyhead[RE,LO]{\thepage} % Oldalszám a külső sarkokban
\fancyfoot[C]{Miskolci Egyetem} % Lábléc közepére az egyetem neve
\renewcommand{\headrulewidth}{0.4pt} % Fejléc alatt választóvonal
\renewcommand{\footrulewidth}{0.4pt} % Lábléc fölé választóvonal

% Saját lista típus definíciója
\newlist{myenum}{enumerate}{6}
\setlist[myenum,1]{label=(\arabic*)}
\setlist[myenum,2]{label=(\alph*)}
\setlist[myenum,3]{label=(\roman*)}
\setlist[myenum,4]{label=(\Alph*)}
\setlist[myenum,5]{label=(\Roman*)}
\setlist[myenum,6]{label=(\arabic*)}

\begin{document}

\title{1. Órai feladat}
\maketitle
\thispagestyle{plain} 
\fancyfoot[RO]{\thepage}
\renewcommand{\footrulewidth}{0.4pt}

% 1. Feladat
\pagestyle{myheadings}
\markboth{Gecse Tamás}{Miskolci Egyetem}

\section{Fejezet címe}
Ez egy példa szöveg, ahol a fejléc páros oldalán a név, páratlan oldalán az egyetem neve jelenik meg.

\newpage

% 2. Feladat
\section{Fejezet 2}
\lipsum[1-4]

\newpage

% 3. Feladat
\section{Egysoros lista}

\newlist{anditemize}{itemize}{1}
\setlist[anditemize,1]{label=--}

\makeatletter
\newcommand{\lastitem}{\@ifnextchar\@endpefalse\relax és }
\makeatother

\begin{anditemize}
    \item elem 1
    \item elem 2
    \lastitem\item elem 3
\end{anditemize}

\newpage

% 4. Feladat
\section{Számozott listák}

\begin{myenum}
    \item Első elem
    \item Második elem
    \begin{myenum}
        \item Beágyazott első elem
        \item Beágyazott második elem
    \end{myenum}
\end{myenum}

\lipsum[5]

\begin{enumerate}
    \item Negyedik elem
    \item Ötödik elem
    \item[ ] Kihagyott számozott elem
\end{enumerate}

\newpage

% 5. Feladat
\section{Leíró lista}

\begin{description}
    \item[] Üres címke
    \lipsum[6]
    \item[Kicsi] Rövid címke
    \lipsum[7]
    \item[Hosszú címke] Egy nagyon hosszú címke, ami akár több sorban is megjelenhet.
    \lipsum[8]
\end{description}

\end{document}
